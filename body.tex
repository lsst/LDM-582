
\section{Scope}

This Working Group is to evaluate and recommend options for lossy compression 
algorithms to be used on LSST images in a way that continues to satisfy LSST 
science use cases. All major types of images shall be considered as candidates 
for compression, including raw data, Processed Visit Images, co-adds, and 
templates.  This working group is to convene immediately and finish its 
investigation by 2017 October 31.

\section{Background}

This group has emerged as the response to \jira{RFC-325} that recognized that user 
experience will be unacceptably impacted by the long latency required to 
access the LSST data from tape media. Unfortunately, preliminary analysis
indicated that retaining all processed images on disk would be too costly 
and therefore not feasible, unless lossy compression is applied. The same
analysis indicated that storing all raw data on disk (w/o lossy compression)
is feasible.

The LSST has traditionally avoided lossy compression for any of its image 
data products (including the large co-added images as well as templates 
retained for each data release). Anecdotal experience from DES and other 
surveys indicates that lossy compression can be applied, without loss of 
scientific fidelity. If this is the case, the reduced disk space needs may 
enable us to retain on low-latency media more data that we otherwise would 
(rather than regenerate or pull from tape. This group has been convened to 
study the problem and report on the results.

The working group should rely as much as possible on prior art found in the
literature, and prefer applications of off-the-shelf solution rather than 
developing custom LSST-specific compression tools.

\section{Responsibilities}

The Working Group (WG) has the following responsibilities:

\begin{itemize}
    \item Define criteria for "science-usable" lossy-compressed processed images across all LSST image types,
    \item Collect compression algorithm candidates (preferring existing in "off-the-shelf” tools/libraries),
    \item Evaluate their compression ratios (at "science-usable" quality),
    \item Evaluate constraints on processing that usage of compression may impose (e.g., avoidance of repeated re-compressions),
    \item Quantify the savings from application of lossy compression, in the context of the LSST Sizing Model (\citedsp{LDM-144}),
    \item Make recommendations on which image types to lossy-compress, the algorithms to apply, and the description of processing constraints these would impose.
\end{itemize}

It should be noted that it is \textbf{not} intended that this group specifically
investigate or recommend file formats (e.g. FITS, HDF5) excepting the utility 
of methods that are available to provide compression within a given format.


\section{Specific Tasks}

\subsection{Acceptable Loss}

The SRD outlines requirements for LSST.  Those requirements should be met 
(or the impact understood) if lossy compression were implemented.  Those 
requirements need to be considered in order to define what is considered 
an ``acceptable loss."

\subsection{Benchmarks}

The Working Group will obtain benchmarks based on existing astronomical data in order 
to provide a measure of lossy compression algorthims': speed, compression factor, 
and fidelity.  Where fidelity means a systematic comparsion of image and catalog 
products with and without lossy compression applied.

\subsection{Recommendations}

The Working Group will provide a recommendation on the suitablity of the implementation
of a lossy compression algorithm of LSST products with a figure of merit that can be 
applied with the LSST Sizing Model (\citedsp{LDM-144}).


\section{Membership}

Membership of roughly four people is optimal and should include persons familiar 
with weak-lensing and difference imaging concerns.
The proposed membership is:

\begin{itemize}
    \item Robert Gruendl (NCSA; \textbf{Chair}),
    \item Paul Price (Princeton),
    \item Bob Armstrong (Princeton),
    \item Krzysztof Findeisen (UW; replacing John Parejko),
    \item Eric Morganson (DES/NCSA; observer)
    \item Ben Emmons (EPO Tucson; observer)
\end{itemize}


\section{Reporting}

The WG Chair shall report directly to the DM Project manager weekly.
